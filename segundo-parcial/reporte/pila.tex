\section{Autómata de Pila}
	\subsection{Descripción del problema}
	Cosas chidas
	\begin{figure}[H]
		\begin{center}
		\includegraphics[width=14cm, height=7cm]{img/pila.png}
		\caption{Representación de un autómata de Pila.}
		\label{fig:diagrama-pila}
		\end{center}
	\end{figure}
	\subsection{Código}
	El código fue realizado en Python 3.5.
	\\Archivo: main\_pila.py
	\begin{lstlisting}[language=Python]
	print('main_pila.py')
	\end{lstlisting}
	\\Archivo: automata\_pila.py
	\begin{lstlisting}[language=Python]
	print('automata_pila.py')
	\end{lstlisting}
	\subsection{Pruebas}
	Pruebas de las opciones del menú.
	\\
	{\large Modo de consola.}
	\begin{figure}[H]
		\begin{center}
			\includegraphics[width=\linewidth, height=6cm]{img/pila-manual.png}
			\caption{Historia del Autómata de Pila.}
			\label{fig:pila1}
		\end{center}
	\end{figure}
	{\large Modo archivo.}
	\begin{figure}[H]
		\begin{center}
			\includegraphics[width=\linewidth, height=20cm]{img/pila-automatico.png}
			\caption{Parte de la historia del Autómata de Pila.}
			\label{fig:pila2}
		\end{center}
	\end{figure}