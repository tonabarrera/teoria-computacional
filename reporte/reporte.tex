\documentclass[12pt, titlepage]{article}
\usepackage[utf8]{inputenc}
\usepackage[spanish]{babel}
\usepackage[letterpaper, margin=2.5cm]{geometry}
\usepackage{lipsum}
\title{Reporte}
\author{Carlos Tonatihu Barrera Pérez}
\date{9 de septiembre de 2016}
\begin{document}
	\maketitle
	\tableofcontents
	\newpage
	
	\section{Alfabeto}
	Un alfabeto es un conjunto de símbolos finito y no vacío. Convencionalmente, utilizamos el símbolo $ \sum $ para designar un alfabeto. \cite{LIBRO}
	\subsection{Descripción del problema}
	El objetivo de esta practica es el generar las potencias del alfabeto binario $ \sum = \lbrace 0, 1 \rbrace $ desde k=0 hasta k = 1000, para después imprimir en un archivo todas las cadenas que se pudieron formar bajo estas condiciones, es decir: 
	\[{\sum}^{+} = {\sum}^{0}\cup{\sum}^{1}\cup{\sum}^{2}\cup\cdots\cup{\sum}^{1000}\]
	Es importante señalar que este conjunto solo es un subconjunto de $ {\sum}^{*} $ que representa todas las cadenas que se pueden formar con este alfabeto binario .
	\subsection{Código}
	Para programar este ejercicio se opto por C ya que nos brinda una gran velocidad a la hora de realizar las operaciones que esta practica demanda.
	\subsection{Pruebas}
	Las pruebas están divididas en modo automático y manual, en ambos dado una k se generan todas las cadenas de longitud 1 hasta k.
	{\large Modo automático.}
	{\large Modo manual.}
	
	\section{Números primos}
	\lipsum[1]
	\subsection{Descripción del problema}
	\lipsum[1]
	\subsection{Código}
	\lipsum[1]
	\subsection{Pruebas}
	\lipsum[1]
	
	\section{AFD Palabras con terminación 'ere'}
	\lipsum[1]
	\subsection{Descripción del problema}
	\lipsum[1]
	\subsection{Diagrama de estados}
	\lipsum[1]
	\subsection{Código}
	\lipsum[1]
	\subsection{Pruebas}
	\lipsum[1]
	
	\bibliography{bibliografia} 
	\bibliographystyle{ieeetr}
\end{document}